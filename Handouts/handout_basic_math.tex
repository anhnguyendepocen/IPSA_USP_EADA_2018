\documentclass[11pt]{article}
\usepackage[utf8]{inputenc}
\usepackage{amsmath}
\usepackage{amssymb}
\usepackage{graphicx}
\usepackage{hyperref}
\usepackage[parfill]{parskip}
\let\oldemptyset\emptyset
\let\emptyset\varnothing


\title{\textbf{Esssentials of Applied Data Analysis\\
				IPSA-USP Summer School 2018}\newline\\
				Basic Mathematics Review}

\author{Leonardo Sangali Barone\\ \href{leonardo.barone@usp.br}{leonardo.barone@usp.br}}
\date{jan/18}

\begin{document}

\maketitle

\section*{Basic Mathematics}

Basic notions of algebra, notation and functions.

\subsection*{Basic Operations}

\begin{tabular}{cll}
	Symbol & Operation & Examples\\
	\hline
	\hline
	$+$ & Addition & $40 + 2 = 42$\\
	$-$ & Subtraction & $44 - 2 = 42$\\
	$\ast or \times$ & Multiplication & $6 * 7 = 42$\\
	$\div$ & Division & $42 \div 7 = 6$\\
	$x^a$ & $x$ to the power of $a$ & $5^2 = 25$; $10^4 = 10.000$\\
	$\sqrt[n]{x}$ & \emph{nth} of $x$ & $\sqrt[2]{25} = 5$; $\sqrt[4]{10.000} = 10$\\
\hline
\end{tabular}
\newline\\

\subsection*{Summation and Product}

Summation: $\sum^n_{i=1}x_i = x_1 + x_2 + ... + x_n$

Example:
\begin{tabular}{cc}

\begin{tabular}{cc}
	$i$ & $x_i$\\
	\hline
	\hline
	1 & 10\\
	2 & 15\\
	3 & 9\\
	4 & 2\\
	5 & 6\\
\end{tabular} & $\sum^5_{i=1}x_i = x_1 + x_2 + x_3 + x_4 + x_5 = 10 + 15 + 9 + 2 + 6 = 42$

\end{tabular}
\newline\\

Product: $\sum^n_{i=1}x_i = x_1 \times x_2 \times ... \times x_n$

Example:
\begin{tabular}{cc}
\begin{tabular}{cc}
	$i$ & $x_i$\\
	\hline
	\hline
	1 & 10\\
	2 & 15\\
	3 & 9\\
	4 & 2\\
	5 & 6\\
\end{tabular} & $\sum^5_{i=1}x_i = x_1 \times x_2 \times x_3 \times x_4 \times x_5 = 10 \times 15 \times 9 \times 2 \times 6 = 16200$
\end{tabular}

\subsection*{Arithmetic Properties}

Associative:
\[(a + b) + c = a + (b + c)\]

Commutative:
\[a + b = b + a\]

Distributive: 
\[a (b + c) = ab + ac\]

Identities:
\[x + 0 = x\]
\[x \times 1 = x\]

Inverse:
\[(-x) + x = 0\]
\[x^{-1} \times x = \left(\frac{1}{x}\right) \times x = 1\]

\end{document}