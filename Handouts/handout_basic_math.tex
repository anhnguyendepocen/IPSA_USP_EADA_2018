\documentclass[11pt]{article}
\usepackage[utf8]{inputenc}
\usepackage{amsmath}
\usepackage{amssymb}
\usepackage{graphicx}
\usepackage{hyperref}
\usepackage[parfill]{parskip}
\let\oldemptyset\emptyset
\let\emptyset\varnothing


\title{\textbf{Esssentials of Applied Data Analysis\\
				IPSA-USP Summer School 2017}\newline\\
				The Basics of Set Theory}

\author{Leonardo Sangali Barone\\ \href{leonardo.barone@usp.br}{leonardo.barone@usp.br}}
\date{jan/18}

\begin{document}

\maketitle

\section*{Basic Mathematics}

Basic notions of algebra, notation and functions.

\subsection*{Operations}

\begin{tabular}{lll}
	$Notation$ & Natural numbers & Examples\\
	$+$ & Addition & $40 + 2 = 42$\\
	$-$ & Subtraction & $44 - 2 = 42$\\
	$*$ & Multiplication & $6 * 7 = 42$\\
	$/$ & Rational numbers & $(..., \frac{-42/13}, 1, \frac{-1/2}, 0, \frac{1/2}, \frac{17/13), \frac{a/b} \forall a, b \in \mathbb{Z}...)$) & all \mathbb{N} are \mathbb{Z}\\
	51 & 52 & 53\\
	61 & 62 & 63\\
	71 & 72 & 73\\
	81 & 82 & 83\\
\end{tabular}



\subsection*{Common Sets}

\begin{tabular}{lll}
	$Notation$ & Natural numbers & Examples & How it relate to other sets\\
	$\mathbb{N}$ & Natural numbers & $(0, 1, 2, ...)$ & \\
	$\mathbb{Z}$ & Integers & $(..., -2, -1, 0, 1, 2, ...)$ & all \mathbb{N} are \mathbb{Z}\\
	$\mathbb{Q}$ & Rational numbers & $(..., \frac{-42/13}, 1, \frac{-1/2}, 0, \frac{1/2}, \frac{17/13), \frac{a/b} \forall a, b \in \mathbb{Z}...)$) & all \mathbb{N} are \mathbb{Z}\\
	51 & 52 & 53\\
	61 & 62 & 63\\
	71 & 72 & 73\\
	81 & 82 & 83\\
\end{tabular}


\subsection*{Set Theory - operations}

	\begin{itemize}
		\item $A \cup B$: union of $A$ and $B$.
		\begin{itemize}
			\item $p \in (A \cup B$): p is an element of $A$ \textbf{OR} $B$.
		\end{itemize}
		\item $A \cap B$: intersection of $A$ an $B$.
		\begin{itemize}
			\item $p \in (A \cup B$): p is an element of $A$ \textbf{AND} $B$.
		\end{itemize}
		\item If $A \cap B$ is equal to $\emptyset$, then A and B are \textbf{disjoint} sets.
		\item $A^c$ ($A'$, \textasciitilde$A$ or simply \emph{not A}) is the set of all elements that does not belong to $A$. $A^c$ is the complement of $A$. 

	\end{itemize}

\subsection*{Venn Diagrams}
 We can represent sets with diagrams. These are called ``Venn Diagrams''. See Figure~\ref{f1} and locate the following sets as a quick exercise:\\

\begin{tabular}{llll}
	1) $A \cup B$ & 5) $(A \cup B) \cup C$ & 9) $A^c$ & 13) $((A \cap B) \cap C)^c$\\
	2) $A \cap B$ & 6) $(A \cap B) \cap C$ & 10) $(A \cap B)^c$ & 14) $((A \cup B) \cap C)^c$\\
	3) $A \cup C$ & 7) $(A \cup B) \cap C$ & 11) $(A \cup C)^c$ &\\
	4) $A \cap C$ & 8) $(A \cap B) \cup C$ & 12) $((A \cup B) \cup C)^c$ &\\
\end{tabular}

\begin{figure}[htp]
\centering
\includegraphics[scale=0.40]{venn.png}
\caption{Venn Diagrams}
\label{f1}
\end{figure}

\end{document}