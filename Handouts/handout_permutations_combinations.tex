\documentclass[11pt]{article}
\usepackage[utf8]{inputenc}
\usepackage{amsmath}
\usepackage{amssymb}
\usepackage{graphicx}
\usepackage{hyperref}
\usepackage[parfill]{parskip}
\let\oldemptyset\emptyset
\let\emptyset\varnothing


\title{\textbf{Esssentials of Applied Data Analysis\\
				IPSA-USP Summer School 2018}\newline\\
				Permutations and combinantions}

\author{Leonardo Sangali Barone\\ \href{leonardo.barone@usp.br}{leonardo.barone@usp.br}}
\date{jan/18}

\begin{document}

\maketitle

\section*{Permutations and combinantions}


	\subsection*{Permutations}

	In situatuations where events have equal probability to happen, it is useful to have a technique to count the number of elements on the sample space. The name of this technique is \emph{permutations}

	For example, consider we have 3 unique objects: A, B and C. In how many unique ways those 3 elements can be arranged in pairs? Answer: {AB, AC, BA, BC, CA, CB}.
	
	How do we compute permutations for a large number of elements? The number of permutations when arranging $k$ elements out of $n$ unique objects is:
	
	\[P_n^k= n \times (n -1) \times (n-2) \times...\times (n-k+2) \times (n-k + 1) = \frac{n!}{(n-k)!} \]

	In the ABC example, $k = 2$ and $n = 3$, thus $P_3^2 = \frac{3!}{(3-2)!} = \frac{3 \times 2 \times 1}{1} = 6$ 		
	
	If we want to arrange $k = 2$ candidates out of $5$ to make a speech (in an specific order) we would have $P_5^2 = \frac{5!}{(5-2)!} = \frac{5!}{3!} = \frac{5 \times 4 \times 3 \times 2 \times 1}{3 \times 2 \times 1} = 5 \times 4 = 20$ .	
	
	\subsection*{Combinations}
	
	\emph{Combinations} are permutations for which the order of the elements does not matter. In our ABC example, AB would be the same combination as BA.

	We compute combinations by dividing the permutation by $k!$:
	
	\[C_n^k = \binom{n}{k} = \frac{P_n^k}{k!}  = \frac{n!}{k!(n-k)!} \]

	If we want to combine $k = 2$ candidates out of $5$ in a debate we would have $P_5^2 = \frac{5!}{2!(5-2)!} = \frac{5!}{2!3!} = \frac{5 \times 4 \times 3 \times 2 \times 1}{2 \ times3 \times 2 \times 1} = 10$ .	

	
\end{document}	
