\documentclass[11pt]{article}
\usepackage[utf8]{inputenc}
\usepackage{amsmath}
\usepackage{amssymb}
\usepackage{graphicx}
\usepackage{hyperref}
\usepackage[parfill]{parskip}
\let\oldemptyset\emptyset
\let\emptyset\varnothing


\title{\textbf{Esssentials of Applied Data Analysis\\
				IPSA-USP Summer School 2018}\newline\\
				The Basics of Set Theory}

\author{Leonardo Sangali Barone\\ \href{leonardo.barone@usp.br}{leonardo.barone@usp.br}}
\date{jan/18}

\begin{document}

\maketitle

\section*{Set Theory}

Basic notions and notation of set theory.

\subsection*{First concepts and notation}

	\begin{itemize}
		\item Sets are a list or collection of objects.
		\item These objects are elements.
		\item $\emptyset$ is the empty set (or null set).
		\item $p \in A$: is p is an element in the set $A$.
		\item $A \subseteq B$: $A$ is a subset of $B$ 
		\item $A \subset B$: $A$ is a proper subset of $B$ and $B$ has at least one element that $A$ does not 
	\end{itemize}

\subsection*{Common Sets}

\begin{tabular}{clll}
	Notation & Meaning & Examples & How it relate to other sets\\
	\hline
	\hline
	$\mathbb{N}$ & Natural numbers & $(0, 1, 2, ...)$ & \\
	$\mathbb{Z}$ & Integers & $(..., -2, -1, 0, 1, 2, ...)$ & all $\mathbb{N}$ are $\mathbb{Z}$\\
	$\mathbb{Z}^{-}$ & Negative Integers & $(..., -2, -1)$ & $\mathbb{Z}^{+}$ is a subset of $\mathbb{Z}$\\
	$\mathbb{Q}$ & Rational numbers & $(..., \frac{-42}{13}, -1, \frac{-1}{2}, 0, \frac{1}{2}, \frac{17}{13}, 100, ...)$ & all $\mathbb{Z}$ are $\mathbb{Q}$\\
	$\mathbb{R}$ & Real numbers & $(..., \frac{-42}{13}, -1, 0, \sqrt{2}, \pi, 100, ...)$ & all $\mathbb{Q}$ are $\mathbb{R}$\\
	$\mathbb{C}$ & Complex numbers & ($1 + 2i$, $42 - 3i$), where $i = \sqrt{-1}$ \\

\end{tabular}

\subsection*{Properties of Sets}

\begin{tabular}{cll}
	Property & Definition & Examples \\
	\hline
	Finite & sets with finite number of elements & $\{S, M, G\}$; $(1, ..., 10)$\\ 
	Infinite & sets with  number of elements & $\mathbb{N}$, $\mathbb{Z}$, $\mathbb{Q}$, $\mathbb{R}$ and $\mathbb{C}$\\
	Countable & number of elements can be counted & $\mathbb{N}$, $\mathbb{Z}$ and $\mathbb{Q}$\\
	Uncountable & not countable & $\mathbb{R}$ and $\mathbb{C}$ \\
	Bounded & finite size or shape (even if infinite)  & x $\in \mathbb{R}: 0 \leq x \leq 1$\\
	Unbounded & infinite size & x $\in \mathbb{R}: x \geq 42$\\
	Ordered & ${a, b, c} \neq {b, a, c}$  & \\
	Unordered & ${a, b, c} = {b, a, c}$  & \\
\end{tabular}


\subsection*{Set Theory - operations}

	\begin{itemize}
		\item $A \cup B$: union of $A$ and $B$.
		\begin{itemize}
			\item $p \in (A \cup B$): p is an element of $A$ \textbf{OR} $B$.
		\end{itemize}
		\item $A \cap B$: intersection of $A$ an $B$.
		\begin{itemize}
			\item $p \in (A \cap B$): p is an element of $A$ \textbf{AND} $B$.
		\end{itemize}
		\item If $A \cap B$ is equal to $\emptyset$, then A and B are \textbf{disjoint} sets.
		\item $A^c$ ($A'$, \textasciitilde$A$ or simply \emph{not A}) is the set of all elements that does not belong to $A$. $A^c$ is the complement of $A$. 
		\item $A \setminus B$ is the set of all elements of set $A$ that does not belong to $B$ (difference). 

	\end{itemize}

\subsection*{Venn Diagramas}
 We can represent sets with diagrams. These are called ``Venn Diagrams''. See Figure~\ref{f1} and locate the following sets as a quick exercise:\\

\begin{tabular}{llll}
	1) $A \cup B$ & 5) $(A \cup B) \cup C$ & 9) $A^c$ & 13) $((A \cap B) \cap C)^c$\\
	2) $A \cap B$ & 6) $(A \cap B) \cap C$ & 10) $(A \cap B)^c$ & 14) $((A \cup B) \cap C)^c$\\
	3) $A \cup C$ & 7) $(A \cup B) \cap C$ & 11) $(A \cup C)^c$ & 15) $A \setminus B$\\
	4) $A \cap C$ & 8) $(A \cap B) \cup C$ & 12) $((A \cup B) \cup C)^c$ & 15) $A \setminus C$\\
\end{tabular}

\begin{figure}[htp]
\centering
\includegraphics[scale=0.40]{venn.png}
\caption{Venn Diagrams}
\label{f1}
\end{figure}

\end{document}